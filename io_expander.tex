% Options here are passed to the article class.
% Most common options: 10pt, 11pt, 12pt
\documentclass[10pt]{datasheet}

% Input encoding and typographical rules for English language
\usepackage[utf8]{inputenc}
\usepackage[english]{babel}
\usepackage[english]{isodate}

% tikz is used to draw images in this example, but you can
% also use \includegraphics{}.
\usepackage{tikz}
\usepackage{pgfplots}
\usepackage{circuitikz}
\usetikzlibrary{calc}

% These define global texts that are used in headers and titles.
\title{IO Expander Datashheet}
\author{Expansion Ltd.}
\date{July 2022}
\revision{Revision 1}
\companylogo{\Huge $\Delta$ Expansion Ltd.}

\begin{document}
\maketitle

\section{Features}

\begin{itemize}
\item{Easy to use and easy to understand I/O Interfacing for any demands}
\item{Powerful plattform for the simplest of requirements}
\item{Giant footprint to improve weight distribution in YOUR product}
\item{Simple UART interfacing for near instant IO operations}
\end{itemize}

\section{Applications}

\begin{itemize}
\item{Super secure systems in family businesses}
\item{Highly specialized Linux systems doing ethernet stuff}
\item{Usage in Hackathons because you can}
\end{itemize}

\section{General Description}
This \textbf{datasheet} enables you to understand the basic concepts of this sophisticated and utterly overdimensioned I/O Expander.


% Switch to next column
\vfill\break

\begin{figure}[h!]
	\centering
    \begin{circuitikz}[european]
        \draw (0,0) node[dipchip,num pins=30 no topmark, external pins width=0.0](C){IC};
		\node [left, font=\tiny] at (C.bpin 1) {SCK};
		\node [left, font=\tiny] at (C.bpin 2) {3.3V};
		\node [left, font=\tiny] at (C.bpin 3) {Ref};
		\node [left, font=\tiny] at (C.bpin 4) {res.};
		\node [left, font=\tiny] at (C.bpin 5) {res.};
		\node [left, font=\tiny] at (C.bpin 6) {res.};
		\node [left, font=\tiny] at (C.bpin 7) {res.};
		\node [left, font=\tiny] at (C.bpin 8) {res.};
		\node [left, font=\tiny] at (C.bpin 9) {res.};
		\node [left, font=\tiny] at (C.bpin 10) {res.};
		\node [left, font=\tiny] at (C.bpin 11) {res.};
		\node [left, font=\tiny] at (C.bpin 12) {5V};
		\node [left, font=\tiny] at (C.bpin 13) {RST};
		\node [left, font=\tiny] at (C.bpin 14) {GND};
		\node [left, font=\tiny] at (C.bpin 15) {VIN};
		\node [right, font=\tiny] at (C.bpin 16) {TX};
		\node [right, font=\tiny] at (C.bpin 17) {RX};
		\node [right, font=\tiny] at (C.bpin 18) {RST};
		\node [right, font=\tiny] at (C.bpin 19) {GND};
		\node [right, font=\tiny] at (C.bpin 20) {D1};
		\node [right, font=\tiny] at (C.bpin 21) {D2};
		\node [right, font=\tiny] at (C.bpin 22) {D3};
		\node [right, font=\tiny] at (C.bpin 23) {D4};
		\node [right, font=\tiny] at (C.bpin 24) {D5};
		\node [right, font=\tiny] at (C.bpin 25) {D6};
		\node [right, font=\tiny] at (C.bpin 26) {D7};
		\node [right, font=\tiny] at (C.bpin 27) {D8};
		\node [right, font=\tiny] at (C.bpin 28) {NC};
		\node [right, font=\tiny] at (C.bpin 29) {NC};
		\node [right, font=\tiny] at (C.bpin 30) {NC};
    \end{circuitikz}
    \caption{Pinout and internal circuit}
\end{figure}

\newpage
% For wide tables, a single column layout is better. It can be switched
% page-by-page.
\onecolumn

\section{Pin Description}
Pin description is done using the table below. (Pinout should really be sufficient though...)

\begin{table}[h]
\begin{threeparttable}
\caption{Pin Description}
\begin{tabularx}{\textwidth}{l | c | X}
    \thickhline
    \textbf{Pin} & \textbf{Name} & \textbf{Function}\\
    \hline
    1  		& SCK 	& Clock \\
    \hline
    2 		& 3.3V 	& 3.3 V Power supply  \\
    \hline
    3		& Ref	& Reference \\
    \hline
    4 - 11 	& Res	& Reserved - Nothing to see here \\
    \hline
    12  	& 5V  	& 5V Power supply \\
    \hline
    13  	& RST 	& Reset on high \\
    \hline
    14  	& GND 	& Ground \\
    \hline
    15  	& VIN 	& Voltage In \\
    \hline
    16  	& TX  	& UART Transmit - Connect to RX of given controller \\
    \hline
    17  	& RX  	& Uart Receive - Connect to TX of given controller \\
    \hline
    18  	& RST 	& Reset on high \\
    \hline
    19  	& GND 	& Ground \\
    \hline
    20  	& D1  	& Digital I/O Pin \\
    \hline
    21  	& D2  	& Digital I/O Pin \\
    \hline
    22  	& D3  	& Digital I/O Pin \\
    \hline
    23  	& D4  	& Digital I/O Pin \\
    \hline
    24  	& D5  	& Digital I/O Pin \\
    \hline
    25  	& D6  	& Digital I/O Pin \\
    \hline
    26  	& D7  	& Digital I/O Pin \\
    \hline
    27  	& D8  	& Digital I/O Pin \\
    \hline
    28  	& NC  	& Not Connected \\
    \hline
    29  	& NC  	& Not Connected \\
    \hline
    30  	& NC  	& Not Connected \\
    \thickhline
\end{tabularx}
\begin{tablenotes}
\item[1]{Errors in pin assignment are users fault.}
\end{tablenotes}
\end{threeparttable}
\end{table}


\newpage
\twocolumn
\section{Detailed Description}

This I/O Expander is a general-purpose input/output (GPIO)
peripheral that provides 8 I/O ports, D1 to D8, controlled through a 
high-speed serial interface. These 8 I/O ports can be used as inputs or
open-drain outputs in any combination.

\subsubsection{Communication}
The communication with this device is done using predefined messages. There are 
three distinct messages to control the I/O Expander and one response message 
which announces the levels of defined input pins. Those concepts are described 
in the next sections.

\subsubsection{Setting Output Ports}
The communication with this device is as simple as it gets. To set or reset I/O 
ports, one must send the SET command and supply 8 bit of data to the device. 
The command to set ports is defined as 0x0F.
The protocol expects the application to identify the ports which shall change 
state to be identified with four LSB according to \ref{table:Pin_address_map}. 
The Mapping is shown in \ref{table:Pin_state_map}.

% Switch to next column
\vfill\break

\subsubsection{I/O direction definition}
Identification of pins as Input / output is done with the I/O definition 
message. One must send the DIR command followed by the corresponding 
The command to specify the direction of ports is defined as 0xF0.
Pins identified with a 1 are set to be output's and pins set to 
0 are set as inputs.

\subsubsection{Reading Inputs}
Similar to the I/O direction definition, the read request message is a special 
case. For this request, command is specified as 0xAA.

When the read request message is received, the I/O Expander returns with a port 
status message. This Message identifies the direction of all ports in the first byte. 
If a port is set as output with high level, the corresponding bit in the second byte 
is set to high (1). If the port is defined as input, the corresponding Bit in the 
second byte represents high state with a high level (1) or low state 
with a low level (0).


\onecolumn

\begin{table}[h]
\begin{threeparttable}
\caption{Pin Address Map}
\label{table:Pin_address_map}
\begin{tabularx}{\textwidth}{l | c | c | c | c | c | c | c | c | X}
    \thickhline
    \textbf{Pin}	& \textbf{bit8}	& \textbf{bit7}	& \textbf{bit6}	& \textbf{bit5}	& \textbf{bit4}	& \textbf{bit3}	& \textbf{bit2}	& \textbf{bit1} 	& \textbf{Hex} \\
    \hline
    D1	&	0	&	0	&	0	&	0	&	0	&	0	&	0	&	1 	& 0xX1 \\
    \hline
    D2	&	0	&	0	&	0	&	0	&	0	&	0	&	1	&	0	& 0xX2 \\
    \hline
    D3	&	0	&	0	&	0	&	0	&	0	&	1	&	0	&	0	& 0xX4 \\
    \hline
    D4	&	0	&	0	&	0	&	0	&	1	&	0	&	0	&	0	& 0xX8 \\
    \hline
    D5	&	0	&	0	&	0	&	1	&	0	&	0	&	0	&	0	& 0x1X \\
    \hline
    D6	&	0	&	0	&	1	&	0	&	0	&	0	&	0	&	0	& 0x2X \\
    \hline
    D7	&	0	&	1	&	0	&	0	&	0	&	0	&	0	&	0	& 0x4X \\
    \hline
    D8	&	1	&	0	&	0	&	0	&	0	&	0	&	0	&	0	& 0x8X \\
\end{tabularx}
\begin{tablenotes}
\item[1]{x means dont care.}
\item[2]{Multiple Pins can be set to high or low at the same time by addressing them at the same time.}
\end{tablenotes}
\end{threeparttable}
\end{table}

\begin{table}[h]
\begin{threeparttable}
\caption{Setting I/O's to high}
\label{table:Pin_state_map}
\begin{tabularx}{\textwidth}{l | c | c | c | c | c | c | c | c | X}
    \thickhline
    \textbf{Pin}	& \textbf{bit8}	& \textbf{bit7}	& \textbf{bit6}	& \textbf{bit5}	& \textbf{bit4}	& \textbf{bit3}	& \textbf{bit2}	& \textbf{bit1} 	& \textbf{Hex} \\
    \hline
    D1	&	0	&	0	&	0	&	0	&	0	&	0	&	0	&	1 	& 0xX1 \\
    \hline
    D2	&	0	&	0	&	0	&	0	&	0	&	0	&	1	&	0	& 0xX2 \\
    \hline
    D3	&	0	&	0	&	0	&	0	&	0	&	1	&	0	&	0	& 0xX4 \\
    \hline
    D4	&	0	&	0	&	0	&	0	&	1	&	0	&	0	&	0	& 0xX8 \\
    \hline
    D5	&	0	&	0	&	0	&	1	&	0	&	0	&	0	&	0	& 0x1X \\
    \hline
    D6	&	0	&	0	&	1	&	0	&	0	&	0	&	0	&	0	& 0x2X \\
    \hline
    D7	&	0	&	1	&	0	&	0	&	0	&	0	&	0	&	0	& 0x4X \\
    \hline
    D8	&	1	&	0	&	0	&	0	&	0	&	0	&	0	&	0	& 0x8X \\
\end{tabularx}
\begin{tablenotes}
\item[1]{x means dont care.}
\item[2]{Multiple Pins can be set to high or low at the same time by addressing them at the same time.}
\end{tablenotes}
\end{threeparttable}
\end{table}
\end{document}
